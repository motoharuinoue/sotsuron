\chapter{あらまし}

本論文では、ファッション分野における画像と感性語、文章の関連性を分析し、また自動コーディネートシステムを提案する。
近年、Convolutional Neural Network(CNN)は画像分類を始めとする様々なタスクで高精度を記録しては画像分類を始めとする様々なタスクで高精度を記録してきた。
そこで本研究では、ファッション分野における画像をCNNに入力することで感性語や文章と対応付け、その関連性を分析した。
さらにその関連性を用いた周辺システムを考案した。
\\
 本研究では、CNNモデルとしてAlexNetとVGG-19モデルを使用した。
これらのモデルにLarge-scale Fashionデータセット(DeepFashion)を用いて事前学習させ、
さらに表現力を高めるため、実際のデータセットを用いてファインチューニングも行った。
感性語はキャプションから抽出した単語とし、教師ラベルとしてCNNを学習させた。
\\
 本論文では、アイテムごと、またアイテムの組み合わせ(コーディネート)の画像に対し、CNNを通して出力される感性語がふさわしいか評価実験を行った。
また反対に、ユーザが欲しいアイテムやコーディネートに関する文を入力した時、ふさわしいアイテムやコーディネート画像が出力されるかについても実験した。
その結果、CNNを通して正しく画像と感性語や文章が対応づけされていることが確認された。
さらに本論文ではこの関連性を用いてユーザが持つアイテムに似合う別のアイテムとコーディネートするシステムを構築した。
