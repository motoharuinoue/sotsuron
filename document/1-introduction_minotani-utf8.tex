¥chapter{はじめに}

漫画は,娯楽としての需要だけでなく,教科書やパンフレットなどに利用される機会も増加している.2013年にカルチュア・コンビニエンス・クラブが16〜69 歳の男女1,252人を対象に実施した「マンガに関するアンケート調査」¥cite{comic2}では,81.7¥%の人が「漫画を読んで人生にプラスの影響を受けた」と回答している.性別,世代別に見てもほぼ偏りのない結果が得られており,日本人のマンガに対する思い入れや親しみがうかがえる.¥¥
 2015年度からスタートした「これも学習マンガだ!〜世界発見プロジェクト〜」¥cite{project}にも注目が集まっている.日本財団により展開されている同プロジェクトでは,マンガの持つ「楽しさ」「わかりやすさ」「共感力」に着目し,社会をより良いものにしていくことを目的としている.同プロジェクトに携わっている日本財団の青柳光昌氏は,「マンガは敷居が低く,子どもから大人まで楽しめる」「視覚に訴えるメディアなので,イメージが湧きやすく,作品の中に入っていきやすい.そうした漫画の特性を生かせば,学習への興味やモチベーションを高めるきっかけになる」と学習マンガの効果について説明している¥cite{comic1}.
¥¥
 このように漫画への注目が集まっている中,漫画表現が学習内容の理解に及ぼす効果も研究されている.向後ら¥cite{kogo}は,マンガによる学習内容の提示が理解と記憶保持に及ぼす影響について検討を行った.その結果,マンガによって表現された教材の学習への利用により,文章だけの表現による場合と比較して長期の記憶保持に効果的であることが予測されている.マンガ表現の持つ面白さや新奇性の効果が学習者の注意をひき,学習への動機付けとなる可能性があると示唆されている.
¥¥
 漫画表現を利用したシステム¥cite{fujimoto, osada, Dharma}も多く提案されている.藤本ら¥cite{fujimoto}は,マンガのコマ割りの技法を導入し,自由な形状とサイズのコマをレイアウトしたプレゼンテーションを作成するツールを提案している.長田ら¥cite{osada}は,ライフログデータを用いて自動的にブログ記事を生成するシステムに漫画表現を取り入れることで,閲覧者に面白さを提供できることを示した.また,Dharmaら¥cite{Dharma}は,SNSを媒体として日本の漫画をコミュニケーション方法として利用するオンラインサービスを提案している.
¥¥
 漫画の表現ではなく漫画の持つエンターテイメント性に焦点を当てた研究も行われている.可児ら¥cite{kani}は,4コマ漫画を利用したCAPTCHAシステムを提案している.同システムでは4 コマ漫画の各コマをランダムに並べ替えて表示し,正しい順序を答えることができた者を人間として判定する.4コマ漫画の持つエンターテイメント性の利用によりユーザは心地よくチューリングテストを受けることが可能である.
¥¥
 また,漫画そのものを自動的に生成する研究¥cite{sakamoto, namiki, Jing}も行われている.坂本ら¥cite{sakamoto}は,個人の日記を漫画形式で自動生成するシステムを,また並木ら¥cite{namiki}は,新聞の記事から漫画を自動的に生成するシステムを提案している.Jingら¥cite{Jing}は会話ビデオを漫画形式に変
¥¥
 家電などの製品においては,紙のマニュアルや電子マニュアルが準備されているものがほとんどである.ユーザは製品を見てすぐには操作方法がわからない場合,マニュアルを読むことは避けられない.しかし,マニュアルは一般的に多量の文字で書かれていることが多く,読みづらく時間がかかるものもある.また,基本機能だけでなく様々なオプション機能に関する説明も多く記載されているため,文書全体の分量が多く,基本的な操作のみを知りたいユーザにとっては煩わしいものであると考えられる.よって,興味を惹き読みやすいマニュアルをユーザに提供する必要があると言える.
¥¥
 マニュアルに関する研究としては,香川ら¥cite{kagawa}や島田ら¥cite{shimada}の研究が挙げられる.
香川らは,WebベースのOSSインストールマニュアル自動生成法の提案を行っている.このシステムでは,設定ファイルの変更を自動的に検知しその変更ログから自動で設定マニュアルを生成する.
島田らは,挿絵が文章理解を促進する効果に対する認知モデルを提案し,挿絵がマニュアルの読解に対して動機付けを高める効果のモデル内での明確化とその評価方法について言及している.
¥¥
 また,マニュアルを漫画化するサービスも行われている.株式会社コミアル¥cite{comial}では,広告やマニュアル,会社案内などの漫画を制作するサービスを行っている.同サービスの利用者の声から,漫画表現の利用によって顧客からの問い合わせ数の増加や分かりやすく簡単な情報伝達の実現が見込まれることがわかる.一方でコミアルでは,白黒マンガの場合1ページあたり27,000円,カラーマンガの場合1ページあたり35,640円の料金が必要であることに加えて,人手で漫画を作成している.そのため,作成に大きなコストが必要であるという問題点がある.よって,読むのが煩わしいマニュアルを低コストで簡単に漫画形式に変換できるシステムの需要が高まっていると考えられる.
¥¥
 そこで,本論文では家電製品などのマニュアルを漫画形式に変換する支援システムを提案する.提案システムでは,ユーザがあらかじめマニュアルから取り出しておいたテキストとそれらの文に紐付けられた画像から漫画を生成する.また,好みのフォントサイズや文体をユーザが選択できるため,各ユーザの目的に合った内容の漫画を生成することが可能である.¥¥
 漫画を生成する際には,背景,キャラクター,吹き出しの画像を重ね,その上にユーザによって選択されたマニュアルの文と画像を表示する.漫画の背景には,マニュアル内のキーワードに対応する場所の画像を基本画像として使用する.また,登場するキャラクターのポーズや表情,吹き出しの種類を多数用意してあるため,多様なデザインの漫画の生成が可能となっている.
¥¥
 以下,第2章では生成漫画の形式と漫画の要素に使用した漫画生成ツールの概要,第3章では提案システムの概要,第4章では評価実験,第5章ではまとめを述べる.
